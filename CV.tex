%%%%%%%%%%%%%%%%%%%%%%%%%%%%%%%%%%%%%%%%%
% This template has been downloaded from:
% http://www.LaTeXTemplates.com
%
% Original author:
% Trey Hunner (http://www.treyhunner.com/)
%  
% 
% Important note:
% This template requires the cv.cls file to be in the same directory as the
% .tex file. The cv.cls file provides the resume style used for structuring the
% document.
%
%%%%%%%%%%%%%%%%%%%%%%%%%%%%%%%%%%%%%%%%%

%----------------------------------------------------------------------------------------
%	PACKAGES AND OTHER DOCUMENT CONFIGURATIONS
%----------------------------------------------------------------------------------------

\documentclass{cv} % Use the custom cv.cls style
\usepackage[dvipsnames]{xcolor}

\usepackage[left=0.45in,top=0.6in,right=0.45in,bottom=0.1in]{geometry} % Document margins
\newcommand{\tab}[1]{\hspace{.2667\textwidth}\rlap{#1}}
\newcommand{\itab}[1]{\hspace{0em}\rlap{#1}}

% Header information
\name{Stefano Sangiovanni}
\role{Ph.D Candidate in Political Science}
\location{Milan, Italy}
\mailto{stefano.sangiovanni@unimi.it}
\github{https://github.com/ste-sangiovanni}

\begin{document}

%----------------------------------------------------------------------------------------
%	EDUCATION SECTION
%----------------------------------------------------------------------------------------

\begin{rSection}{Education}

\begin{rSubsection}{NASP Graduate School, University of Milan}{October 2023 -- Present}{Ph.D Candidate in Political Studies}{Milan, Italy}
\item Supervisors: Prof. Andrea Ceron, Prof. Giovanna Invernizzi
\item Thesis Title (provisional): "The balance between positional and valence issues in party competition"
\item Research Interests: Political Parties, Computational Methods, Electoral Behavior, Political Scandals, Quantitative Text Analysis, Survey Experiments
\end{rSubsection}

\begin{rSubsection}{University of Milan}{October 2020 -- March 2023}{Master's Degree in Administration and Public Policy, 110/110 with honours}{Milan, Italy}
\item Thesis Title: "Communicating in times of crisis: a comparative analysis of school policy discourse in Italy and Great Britain during the covid-19 pandemic"
\item Conducted a comparative analysis of 14 school policy discourses in Italy and Great Britain during the COVID-19 pandemic, building a framework to measure the quality of the discourse. Utilized sentiment analysis with Python on over 15k tweets to assess whether high-quality discourse resulted in a positive sentiment from users
\end{rSubsection}

\begin{rSubsection}{University of Milan-Bicocca}{October 2017 -- October 2020}{Bachelor's Degree in Sociology}{Milan, Italy}
\item Thesis Title: Gender Representations in the Media: The Model of the Global Media Monitoring Project
\end{rSubsection}

\end{rSection}

%----------------------------------------------------------------------------------------
%	TRAINING SECTION
%----------------------------------------------------------------------------------------

\begin{rSection}{Summer, Winter Schools \& Additional Training}

    \begin{rTrainingSubsection}{Large Language Models for Political Scientists, SISP 2024}{September 2024}
    {Gained hands-on experience with LLMs for political science research using Python}
    \end{rTrainingSubsection}
    
    \begin{rTrainingSubsection}{Summer Institute in Computational Social Science, University of Bologna}{June 2024}
    {2-weeks summer school on computational methods, with competitive selection process}
    \end{rTrainingSubsection}
    
    \begin{rTrainingSubsection}{RECSM Winter Methods School, Universitat Pompeu Fabra}{March 2023}
    {Network Analysis and Visualization with R}
    \end{rTrainingSubsection}
    
    \begin{rTrainingSubsection}{RECSM Summer Methods School, Universitat Pompeu Fabra}{June 2022}
    {Introduction to Python, Introduction to R, Survey Methodology and Survey Design} 
    \end{rTrainingSubsection}
    
\end{rSection}
     
%----------------------------------------------------------------------------------------
% TEACHING SECTION
%----------------------------------------------------------------------------------------

\begin{rSection}{Teaching \& Tutoring}

% Subsection: Guest Lectures
\subtitle{Guest Lectures}

\begin{rTeachingSection}{Political Sociology}{2023 -- 2024}{BA in International Sciences and European Institutions, University of Milan}\\ 
Course held by Prof. Cristiano Vezzoni \\    
Lecture: "Measuring Democracy: Theory and Practice" -- \href{https://github.com/ste-sangiovanni/democracy_seminar}{Lecture slides}   
\end{rTeachingSection}
\begin{rTeachingSection}{Politics, Philosophy and Public Affairs}{2024}{Thesis Seminars for Master's students, University of Milan} \\ 
Lecture: "How to plan a social sciences research design" -- \href{https://github.com/ste-sangiovanni/PPPA_Thesis_Seminar}{Lecture slides}  \\ 
Lecture: "Introduction to Qualitative Research" -- \href{https://github.com/ste-sangiovanni/PPPA_Thesis_Seminar}{Lecture slides}   
\end{rTeachingSection}
    
% Subsection: Tutoring
\subtitle{Academic Tutoring \& Teaching Assistant}

\begin{rTeachingSection}{Master’s Degree Course in Politics, Philosophy and Public Affairs}{January 2024  -- Present}{University of Milan, Italy}
\end{rTeachingSection} 
\begin{rTeachingSection}{Bachelor’s Degree Course in Political Science}{September 2023 -- Present}{University of Milan, Italy}
\end{rTeachingSection} 
\end{rSection}
    
\newpage
%----------------------------------------------------------------------------------------
%	PUBLICATIONS AND CONFERENCES SECTION
%----------------------------------------------------------------------------------------

\begin{rSection}{Publications, Conferences \& Seminar Presentations}
\subtitle{Conferences \& Seminar Presentations}

\begin{rPresSection}{2024}{Platini, U. \& \textbf{Sangiovanni, S.}, “Analyzing Public Perceptions of Government Actions and Political Scandals Through Survey Experimental Designs.” \textit{POLS PIRSS Seminars Series}, University of Milan}
\end{rPresSection}
\begin{rPresSection}{2024}{SISP 2024 Conference. \textit{Italian political science association, University of Trieste.} 12-14 September 2024}
\end{rPresSection}
\begin{rPresSection}{2024}{\textbf{Sangiovanni, S.}, “Valence Theory and Political Scandals: A Proposal of a Conjoint Experiment.” \textit{CCA Days}, Collegio Carlo Alberto, Turin}
\end{rPresSection}
\begin{rPresSection}{2022}{\textbf{Sangiovanni, S.}, Lombardi, A., Nascone, M., Rossoni, E. “European Emission Trading System: The impact of EU-ETS on costs and benefits.” \textit{Jean Monnet Centres of Excellence}, University of Milan}
\end{rPresSection}

\subtitle{Working Papers}

\begin{rWPSection}       
{\textbf{Sangiovanni, S.} “Policy Shifts and Strategic Valence Emphasis in Electoral Competition”} 
\end{rWPSection}
\begin{rWPSection}
Celebrin, G., {\textbf{Sangiovanni, S.} "Generational Perspectives on Democracy: Migrants’ Satisfaction Across National
and European Levels"}    
\end{rWPSection}    
\begin{rWPSection}    
{Holdrich, E., \textbf{Sangiovanni, S.}, Végh, M. “Analyzing Hungarian Parliamentary Debates with ManifestoBERTa”} 
\end{rWPSection}
\begin{rWPSection}
{\textbf{Sangiovanni, S.} “Valence Theory and Political Scandals: a Conjoint Experiment Design”} 
\end{rWPSection}
\begin{rWPSection}
{\textbf{Sangiovanni, S.} “Feeding the Cycle of Negativity: Exploring the Media's Influence on the Tone of Electoral Campaigns”}
\end{rWPSection}
\end{rSection}
%----------------------------------------------------------------------------------------
%	PROFESSIONAL EXPERIENCE SECTION
%----------------------------------------------------------------------------------------

\begin{rSection}{Relevant Professional Experiences}

    \subtitle{Academic}

    \begin{rExperienceSection}{Conference Organization Team Member}{September 2023}{ESPAnet 2023, University of Milan, Italy}
    Managed logistics for conference sessions and collaborated with colleagues in supervising students responsible for assisting conference attendees. Contributed to the preparation of conference materials and documentation
    \end{rExperienceSection}
    
    \begin{rExperienceSection}{Research Assistant}{October 2020 -- December 2020}{Global Media Monitoring Project, University of Milan-Bicocca, Italy}
    Collected and analyzed tweets from 7 Italian newspapers as part of the Global Media Monitoring Project, in collaboration with the Milano-Bicocca University research group. \href{https://www.osservatorio.it/wp-content/uploads/2016/03/GMMP-2020-Nationa-Report-Italy.pdf}{Read the italian GMMP 2020 Report}
    \end{rExperienceSection}
    
    \begin{rExperienceSection}{Tutoring Online Project}{October 2022 -- March 2023}{Bocconi University - Harvard University, Milan, Italy}
    Delivered over 40 hours of tutorship in Mathematics, English, and Italian to support students with financial and learning difficulties
    \end{rExperienceSection}
    
    \subtitle{Extra-Academic}
    
    \begin{rExperienceSection}{Esports Caster -- Content Creator}{September 2014 -- May 2018}{Spaziogames, Milan, Italy}
    Created content and managed live streaming events via OBS Studio, provided live commentary for esports tournaments with over 5k viewers and wrote articles and reviews for the website
    \end{rExperienceSection}

\end{rSection}

%----------------------------------------------------------------------------------------
%	SKILLS SECTION
%----------------------------------------------------------------------------------------

\begin{rSection}{Skills}

\begin{skillsTable}
Languages & English (IELTS Academic Band 8, C1), Italian (native), Spanish (A2) \\
Programming Languages & R, Python, LaTeX, SQL \\
Data Analysis Tools & STATA, Tableau, SPSS \\
Developer Tools & Git, GitHub, VisualStudio Code, Jupyter Notebook \\
Live Streaming Tools & OBS Studio, OBS Ninja, StreamElements \\
GIS Tools & QGIS
\end{skillsTable}

\end{rSection}

%----------------------------------------------------------------------------------------
%	FUNDING AND AWARDS SECTION
%----------------------------------------------------------------------------------------



\end{document}
